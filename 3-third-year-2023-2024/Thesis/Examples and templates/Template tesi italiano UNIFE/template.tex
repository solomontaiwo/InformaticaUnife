\documentclass[a4paper, 12pt]{book}
\usepackage{amssymb}
\usepackage[english,italian]{babel}
\usepackage{xspace}
\usepackage{tikz}
\usepackage[newitem,newenum,neverdecrease]{paralist}
\usepackage{amsmath,amssymb,amsfonts,mathrsfs,latexsym,stmaryrd}
\usepackage{mathtools}
\usepackage{amsthm}
\usepackage{ifpdf}
\usepackage{cases}
\usepackage{ragged2e} 	% https://tex.stackexchange.com/questions/89680/how-can-one-set-full-justification-within-left-justified-raggedright-text
\usepackage{breakcites} % https://tex.stackexchange.com/questions/2773/how-do-i-make-latex-push-long-citations-to-a-new-line
\usepackage{hyperref}	% http://tex.stackexchange.com/questions/73862/how-can-i-make-a-clickable-table-of-contents
\usepackage{fancyhdr}	% https://www.sharelatex.com/blog/2013/08/06/thesis-series-pt2.html
\usepackage{titlesec} 	% http://tex.stackexchange.com/questions/11444/how-to-format-the-chapter-heading
\usepackage[T1]{fontenc}
\usepackage[utf8]{inputenc}
\usepackage{xcolor}
\usepackage{geometry}
\usepackage{layout}		% http://tex.stackexchange.com/questions/50258/margins-of-book-class
\usepackage{pseudocode}
\usepackage{chngcntr}


% fixes for pseudocode-hyperref interaction
\counterwithout{pseudocode}{section}
%\counterwithin{pseudocode}{chapter}
\renewcommand{\thepseudocode}{\thechapter.\arabic{pseudocode}}

\newcounter{newpseudonum}[pseudocode]
\renewcommand{\thenewpseudonum}{\thepseudocode\alph{newpseudonum}}
\providecommand{\labelline}[2][0.5in]{&\refstepcounter{newpseudonum}\hspace*{#1}\mbox{(\thenewpseudonum)}\label{#2}}
\ifpdf
  \providecommand{\refline}[1]{\hyperref[#1]{(\ref*{#1})}}
\else
  \providecommand{\refline}[1]{\ref*{#1}}
\fi
\newcommand{\pcodetab}[1]{\hspace*{#1ex}}

% alignement fixes for pseudocode blocks
\renewcommand{\DO}{\\\!\!\pcodetab{1}\mbox{ \bfseries \makebox[0pt][l]{do}\phantom{xx} }}
\renewcommand{\ELSE}{\\\pcodetab{1}\mbox{ \bfseries \makebox[0pt][l]{else}\phantom{then} }}
\renewcommand{\RETURN}[1]{\ifthenelse{\equal{#1}{} }{\mbox{\bfseries return}}{\mbox{\bfseries return}#1}}
\newcommand{\FUNCTION}[2]{\mbox{\bfseries proc }\mbox{\textsc{#1}}\left(\ensuremath{#2}\right)\\}
\newcommand{\ENDFUNCTION}{}
\newcommand{\FCALL}[2]{\mbox{\textsc{#1}}\left(\ensuremath{#2}\right)\\}



% code box length
\newlength{\pcodewidth}
\setlength{\pcodewidth}{\textwidth}
\addtolength{\pcodewidth}{-58pt}

% code environment
\newenvironment{code}[1]{
\begin{Sbox}
\!\!\begin{minipage}{#1}%\pcodewidth}
\bfseries
\noindent
%\begin{math}
\small
$$
\begin{array}{@{\hspace*{1ex}}lr@{}}
}{
\end{array}
$$
%\end{math}
\end{minipage}\vspace{-2mm}
\end{Sbox}
\shadowbox{\TheSbox}{}
}	% for pseudocodes

%%% Formattazione del header del Capitolo
\titleformat
{\chapter} % command
[display] % shape
{\bfseries\Huge\itshape} % format
{\flushright \color{black!45}{\Large Capitolo \thechapter}} % label
{0.5ex} % sep
{
    \rule{\textwidth}{3pt}
    \vspace{1ex}
    \centering
} % before-code
[
\vspace{-0.5ex}%
\rule{\textwidth}{3pt}
] % after-code
%%% end

%%% Formattazione dei hyper riferimenti (hyperref)
% hyperref setup -- http://tex.stackexchange.com/questions/73862/how-can-i-make-a-clickable-table-of-contents
\definecolor{Aqua}{rgb}{0,0.45,0.45}
\definecolor{Gold}{rgb}{0.55,0.55,0}
\hypersetup{
    bookmarks=true,         % show bookmarks bar?
    unicode=false,          % non-Latin characters in Acrobat’s bookmarks
    pdftoolbar=true,        % show Acrobat’s toolbar?
    pdfmenubar=true,        % show Acrobat’s menu?
    pdffitwindow=false,     % window fit to page when opened
    pdfstartview={FitH},    % fits the width of the page to the window
    pdftitle={Analisi e strutturazione di dati storici medici densitometrici per fini statistici},    % title
    pdfauthor={Bombonati Stefano},     % author
    pdfsubject={Analisi e strutturazione di dati},   % subject of the document
    pdfcreator={Bombonati Stefano},   % creator of the document
    pdfproducer={Bombonati Stefano}, % producer of the document
    pdfkeywords={Analisi e strutturazione di dati medici densitometrici}, % list of keywords
    linktocpage=false,		% number hyperref'd =true, false otherwise
    pdfnewwindow=true,      % links in new PDF window
    colorlinks=true,       % false: boxed links; true: colored links
    linkcolor=Aqua,          % color of internal links (change box color with linkbordercolor)
    citecolor=magenta,        % color of links to bibliography
    filecolor=green,      % color of file links
    urlcolor=Gold           % color of external links
}
%%% end

% https://tex.stackexchange.com/questions/95488/list-of-figures-and-page-numbering
\makeatletter
\newcommand{\emptypage}[1]{%
  \cleardoublepage
  \begingroup
  \let\ps@plain\ps@empty
  \pagestyle{empty}
  #1
  \cleardoublepage}
\makeatletter

% base line strech (default 1.0) -- interlinea
\renewcommand{\baselinestretch}{1.2}

\begin{document}
%%% Nuova geometria per la pagina del titolo
% visto che la classe del documento è book, le pagine pari e dispari avranno geometrie diverse,
% mentre quella del titolo deve essere unica!
\newgeometry{
  top=2cm,
  bottom=2.5cm,
  left=2.5cm,
  right=2.5cm,
  headsep=25pt,
  headheight=14.5pt
}

%%% Pagina del titolo
\begin{titlepage}
	\topskip0pt
	%\vspace*{\fill}
	\centering
	\vspace*{20mm}
	\includegraphics[]{logo.png}\\
	\vspace*{1cm}
	\huge \textbf{\textsc{Università degli Studi di Ferrara}}
	\Large \textsc{Corso di Laurea in Informatica}
	
	\vspace*{1.5cm}
	\hrule width \hsize \kern 1mm \hrule width \hsize height 2pt
	\vspace*{10mm}
	\Huge \emph{\textbf{Il mio titolo della tesi in italiano}}
	\vspace*{10mm}
	\hrule width \hsize height 2pt
	\vspace*{1mm}
	\hrule width \hsize \kern 1mm
	
	\vspace*{15mm}
	\begin{minipage}{0.45\textwidth}
		\begin{flushleft} \Large
			\emph{Relatore:}\\
			\Large \textbf{Prof. Nome \textsc{Cognome}}
		\end{flushleft}
	\end{minipage}
	\begin{minipage}{0.45\textwidth}
		\begin{flushright} \Large
			\emph{Laureando:} \\
			\Large \textbf{Nome \textsc{Cognome}}
		\end{flushright}
	\end{minipage}
	
	\vspace*{20mm}
	\Large \textsc{Anno Accademico $2016-2017$}
\end{titlepage}
\restoregeometry

\addtocontents{toc}{~\hfill\textbf{Page}\par}	% https://texblog.org/2011/09/09/10-ways-to-customize-tocloflot/
\pagestyle{empty}
\clearpage
\tableofcontents
\thispagestyle{empty}
\addtocontents{toc}{\protect\thispagestyle{empty}}	% http://tex.stackexchange.com/questions/2995/removing-page-number-from-toc

\chapter{Introduzione}
%%% Fancy header settings, queste impostazioni vanno fatte solo una volta all'inizio del primo capitolo!
\pagestyle{fancy}
\fancyhf{}
\renewcommand{\headrulewidth}{2pt}
\fancyhead[EL]{\textbf{\textsf{\nouppercase\thepage}}}
\fancyhead[ER]{\textbf{\textsf{\nouppercase\leftmark}}}
\fancyhead[OR]{\textbf{\textsf{\nouppercase\thepage}}}
\fancyhead[OL]{\textbf{\textsf{\nouppercase {\rightmark}}}}
%%% end

%%% all'inizio di ogni capitolo, questa impostazione rimuove il numero di pagina, provare a commentare per vedere la differenza
\thispagestyle{empty}

Scrivi qua l' introduzione
\newpage
Qua inizia la numerazione delle pagine, guardare in alto della pagina

\begin{figure}[t]
\begin{center}
\includegraphics[scale=0.7]{logo.png}
\end{center}
\caption{<===Diventa in italiano.}\label{fig:uni}
\end{figure}


\end{document} 